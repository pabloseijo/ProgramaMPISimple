\documentclass[conference, a4paper]{IEEEtran}

\usepackage{cite}
\usepackage{amsmath,amssymb,amsfonts}
\usepackage{algorithmic}
\usepackage{graphicx}
\usepackage{textcomp}
\usepackage{xcolor}
\usepackage{hyperref} % Para enlaces y referencias cruzadas
\usepackage{fancyhdr} % Para controlar los encabezados y pies de página

% Agregar números de página
\pagestyle{plain}

\begin{document}

\title{Título del Paper}
\author{
    \IEEEauthorblockN{Tu Nombre}
    \IEEEauthorblockA{\\Departamento de Investigación\\
    Nombre de la Institución\\
    Ciudad, País\\
    Email: tu.email@ejemplo.com}
}
\date{\today}

\maketitle

\begin{abstract}
En este resumen se proporciona una visión general de los principales objetivos, métodos y hallazgos de la investigación. El resumen debe ser conciso, entre 150 y 250 palabras, y ofrecer una descripción clara del propósito y los resultados de la investigación.
\end{abstract}

% Agregar las palabras clave
\begin{IEEEkeywords}
Palabras clave, IEEE, formato, investigación científica.
\end{IEEEkeywords}

\section{Introducción}
En la sección de introducción, debes presentar el contexto y la importancia de tu investigación. Aquí es donde justificas la relevancia del problema abordado, explicas el estado del arte, y defines claramente el objetivo del trabajo. Se recomienda hacer referencias a trabajos previos relacionados con tu investigación \cite{referencia1}.

\section{Metodología}
Esta sección describe los métodos y procedimientos empleados en tu investigación. Detalla cómo se llevaron a cabo los experimentos, análisis o simulaciones, y qué herramientas o técnicas se utilizaron para alcanzar los resultados. También debes explicar por qué elegiste esos métodos y su relación con los objetivos de la investigación.

\section{Resultados}
En esta sección se presentan los resultados obtenidos durante el desarrollo de tu investigación. Se pueden usar tablas, gráficos y descripciones detalladas para mostrar los hallazgos de manera clara. Asegúrate de que los resultados sean reproducibles y que los datos estén organizados adecuadamente.

\begin{figure}[htbp]
\centerline{\includegraphics[width=0.5\textwidth]{figura_ejemplo.png}}
\caption{Descripción de la figura.}
\label{figura1}
\end{figure}

\section{Conclusiones}
En las conclusiones, se hace un resumen de los principales hallazgos de tu trabajo. Además, puedes incluir posibles aplicaciones de los resultados, las limitaciones de la investigación y las recomendaciones para trabajos futuros.

\section*{Agradecimientos}
Esta sección es opcional y se usa para agradecer a las personas o instituciones que contribuyeron de alguna forma al desarrollo de la investigación, como financiamiento o asistencia técnica.

\bibliographystyle{IEEEtran}
\bibliography{bibliografia}

\end{document}
